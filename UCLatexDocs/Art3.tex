% ----------------------------------- PORTADA -----------------------------------
\begin{titlepage}
    \begin{center}
        \vspace*{1cm}
        
        \includegraphics[width=0.4\linewidth]{figuras/cuenca.png}\\
        \LARGE
        \unidad\\
        \programa\\
        \curso
        
        \vspace{1cm}
        
        \Huge
        \textbf{\titulo}
            
        \vspace{0.5cm}
        \LARGE
        \subtitulo
            
        \vspace{1.5cm}
        
        \large    
        \autores
            
        \vfill
        
        \lugar\\
        \fecha
        
    \end{center}
\end{titlepage}

% -------------------------------DOCUMENTO PRINCIPAL -------------------------------
\documentclass[journal]{IEEEtran}
\usepackage[utf8]{inputenc}
\usepackage[spanish]{babel}
% Soporte para buenas urls e hipervínculos entre secciones
\usepackage{hyperref}
% Citas y referencias en formato APA
% Si quiere las citas y referencias en IEEE comente esta línea
\usepackage{apacite}
\usepackage{graphicx}
% Código fuente con números de línea
\usepackage{listings}
\usepackage{float}
% Puede cambiar el lenguaje de código fuente
% https://www.overleaf.com/learn/latex/code_listing#Supported_languages
\lstset{
    language=C,
    basicstyle=\footnotesize,
    numbers=left,
    stepnumber=1,
    showstringspaces=false,
    tabsize=1,
    breaklines=true,
    breakatwhitespace=false,
}

%====================================================%
\def \unidad{Universidad de Cuenca}
\def \programa{Ingeniería en Telecomunicaciones}
\def \curso{Grupo \# 3 - Cálculo en varias variables}
\def \titulo{Hola}
\def \subtitulo {Resolución del ejercicio planteado en clase}
\def \autores{
    Javier Alexander Procel Morocho\\
    javier.procel@ucuenca.edu.ec\\
    
    \vspace{0.5cm}
    
    Mateo Feijo\\
    Correo electrónico\\

    \vspace{0.5cm}
    
    Julio Josué Macas Guamán\\
    julio.macas@ucuenca.edu.ec\\
}
\def \fecha{Octubre 2023}
\def \lugar{
    Azuay, 
    Cuenca
}

%====================================================%
\begin{document}
\input{portada}
\tableofcontents

%====================================================%
\section{Ejercicios}\label{intro}

Resolver los ejercicios 5, 8, 19, 24, 37 y 38 del bloque de ejercicios 13.2
%====================================================%
\section{Problema y solución}\label{antecedentes}

\subsection{Ejercicio 5-8}
 
   \item (a) Trace la curva en un plano con la ecuación vectorial dada
   \item (b) Determine r’(t).
   \item (c) Trace el vector de posición r(t) y el vector tangente r9(t) para
el valor dado de t. \vspace{3mm}
 
   \item 5) $r(t)= e^{2t}i+ e^tj \hspace{1cm}    t= 0$ \vspace{3mm}

a) 
\begin{figure} [H]
    \centering % Opcional, para centrar la imagen
    \includegraphics[width=0.5\textwidth]{5.1.png}
    \caption{Gráfica ejercicio 5}
    \label{fig:etiqueta}
\end{figure}
b) \vspace{1mm}
\begin{equation}
    \centering
    r'(t)= \frac{\mathrm{d} }{\mathrm{d} t}(e^{2t})i+ \frac{\mathrm{d} }{\mathrm{d} t}(e^t)j\vspace{1mm} 
\end{equation}



\begin{equation}
    \centering
    r'(t)= 2e^{2t}i+ e^tj \vspace{2mm} 
\end{equation}
\newpage
c) 
\begin{figure} [H]
    \centering % Opcional, para centrar la imagen
    \includegraphics[width=0.5\textwidth]{5}
    \caption{Gráfica de la curva ejercicio 5}
    \label{fig:etiqueta}
\end{figure}

\item 8) r (t)= (cos (t) +1)i+ (sen(t)-1)j \hspace{1cm} t= -$\pi$ / 3

a) 

\begin{figure} [H]
    \centering % Opcional, para centrar la imagen
    \includegraphics[width=0.5\textwidth]{8.1.png}
    \caption{Gráfica de la curva ejercicio 8}
    \label{fig:etiqueta}
\end{figure}

b) 
\begin{equation}
    \centering
    r'(t)= \frac{\mathrm{d} }{\mathrm{d} t}(cos(t)+1)i+ \frac{\mathrm{d} }{\mathrm{d} t}(sen(t)-1)j\vspace{1mm} 
\end{equation}



\begin{equation}
    \centering
    r'(t)= -sen(t)i + cos(t)j 
\end{equation} \newpage

c) 
\begin{figure} [H]
    \centering % Opcional, para centrar la imagen
    \includegraphics[width=0.5\textwidth]{8.png}
    \caption{Gráfica de vectores ejercicio 8}
    \label{fig:etiqueta}
\end{figure}
%====================================================%

\subsection{Ejercicio 17-20}
\item Determine el vector tangente unitario T(t) en el punto con
el valor dado del parámetro t.
\item 19)
\begin{equation}
    r(t) = cos(t)i + (3t)j + 2sen(2t)k, t = 0
\end{equation}
Encontramos la derivada
\begin{equation}
    r´(t) = -sen(t)i + 3j + 4cos(2t)k
\end{equation}
Dado el valor de t = 0, tenemos
\begin{equation}
    r(0) = i
\end{equation}
\begin{equation}
    r´(0) = 3j + 4k
\end{equation}
El vector tangente unitario en el punto (1,0,0) es
\begin{equation}
    T(0) = \frac{r´(0)}{|r´(0)|}
\end{equation}
\begin{equation}
    T(0) = \frac{3j + 4k}{\sqrt{3^2 + 4^2}}
\end{equation}
\begin{equation}
    T(0) = \frac{3j + 4k}{5}
\end{equation}
\begin{equation}
    T(0) = (\frac{3}{5})j + (\frac{4}{5})k
\end{equation}

%====================================================%
\subsection{Ejercicios 23-26}
\item Determine ecuaciones paramétricas para la recta tangente
a la curva con las ecuaciones paramétricas dadas en el punto
especificado
\item 24)
\begin{equation}
    x = ln(t + 1), y = tcos(2t), z = 2^t, (0,0,1)
\end{equation}
\begin{equation}
    r(t) = (ln(t + 1), tcos(2t), 2^t)
\end{equation}
Encontramos la derivada
\begin{equation}
    r´(t) = (\frac{1}{t + 1}, cos(2t) - 2t*sen(2t), 2^t*ln2)
\end{equation}
El vector parametro correspondiente al punto (0,0,1) es t = 1, asi que
\begin{equation}
    r´(1) = (\frac{1}{2}, 1.06, 1.38)
\end{equation}
La recta tangente que pasa por (0,0,1) es paralela al vector (\frac{1}{2}, 1.06, 1.38). Entonces las ecuacione spara

%====================================================%
\subsection{Ejercicios 37-38}\label{intro}

\textbf{Evalúe la integral de:}

\[\int_{0}^{1}(\frac{1}{t+1}i,\frac{1}{t^{2}+1}j,\frac{t}{t^{2}+1}k)dt\]

Lo resolvemos por partes:
Para i:
\[\int_{0}^{1}(\frac{1}{t+1})dt\]
\[u = t+1\]
\[du = dt\]
\[\int_{0}^{1}(\frac{1}{u}i)du = ln(u) = ln(t+1)\]
\[ln((1)+1) - ln(0)+1) = ln(2)-ln(1) = 0.69\]

Para j:
\[\int_{0}^{1}(\frac{1}{t^{2}+1})dt = arctang(t)\]
\[arctang(1) - arctang(0) = \frac{\pi}{4}\]

Para k:
\[\int_{0}^{1}(\frac{t}{t^{2}+1})dt\]
\[u = t^{2}+1\]
\[du = 2tdt\]
\[\frac{1}{2} \int_{0}^{1}(\frac{du}{u}) = \frac{1}{2}ln(u) = \frac{1}{2}ln(t^{2}+1)\]
\[\frac{1}{2}ln((1)^{2}+1)-\frac{1}{2}ln((0)^{2}+1) = \frac{1}{2}ln(2)-\frac{1}{2}ln(1) = 0.35\]

Para el segundo ejercicio seria: 
\[\int_{0}^{\frac{\pi }{4}}(sec(t)tan(t)i, tcos(2t)j, sen^{2}(2t)cos(2t)k)dt\]

%====================================================%
\end{document}
